\section{Introduction}
\label{sec:intro}

The business-model of many major consumer-facing Cloud providers like Dropbox consists in storing huge volumes of data. Customers expect a high level of reliability, but rarely envisage paying top dollar for such services.\footnote{The hidden costs of these free services are left for another time.} 
The very-large scale of Cloud provider setups implies that failures occur fairly often \autocite{failures-study}, thus fault-tolerance is intrinsically built into the design of large-scale systems. 
The result is that irrecoverable user data loss caused by technical failures is rare\autocite{racs}.

To offer that kind of guarantees towards their customers, Cloud providers need to implement techniques providing redundancy, while keeping costs under control. 
The long-established way of providing redundancy is replication, which consists in duplicating functionality across components that are unlikely to fail simultaneously. 
For data storage, this means storing multiple replicas in geographically distinct locations. 
The cost of storage is multiplied by the amount of redundancy wanted. 
To store data on a redundant array, the system has to send write requests to each component in the array. 
When a copy of the data is unavailable, the system can simply query another copy. 
A system that is replicated $n$ times allows $n-1$ simultaneous failures.

An alternative solution to provide data redundancy is erasure coding. 
Error-correcting codes can decrease the storage overhead compared to replication, but recovering errors requires decoding. 
Furthermore, traditional erasure codes, while being storage-efficient, were not designed to handle failures occurring on distributed storage systems. 
Replacing a failed disk requires the decoding of all the data objects with a block on it. With a $(k,n-k)$ maximum distance separable (MDS)\autocite{macwilliams1977theory} code like a Reed-Solomon code (RS), the required bandwidth and latency to fetch $k$ geographically distributed blocks required for decoding is significant, and using such codes on geographically distributed data centers would saturate the network links. 
This observation has led to the study of codes that can repair failed blocks by using less information than required to recover the original data. 
The drawback of these \emph{locally repairable codes} (LRCs)\autocite{papailiopoulos2014locally} is a lower code rate, thus they are a tradeoff between MDS codes and replication.

%The tradeoffs between local repairability and storage overhead have been extensively studied over the last few years [2], [3].

%  and consists in transforming original data into a longer, fault-tolerant form. Erasure coding can provide the same level of reliability as replication with much less storage overhead \autocite{Weatherspoon2002}.
% Computational cost is one of the drawbacks of erasure coding.
% Every write to an erasure-coded system requires computing parity blocks.
% When reading data back from a replicated data storage system, unavailable blocks can be reconstructed using a combination of certain available blocks.
% Reading degraded erasure-coded data implies retrieving a larger amount of blocks than would be needed to read intact data. Also, some computations need to be performed on the blocks retrieved in order to recreate the unavailable blocks.
% Another drawback that modern erasure coding algorithms try to overcome is the high amount of data required to repair faulty blocks.
% Traditional erasure coding algorithms such as \ac{rs} need to decode a complete stripe in order to repair a single faulty block in that stripe.
% It has been calculated that if Facebook coded \SI{50}{\percent} of the data in one of their clusters, its network would be completely saturated~\autocite{XorbasVLDB}.


% The newer generation of erasure codes focuses on reducing the amount of data needed for repairing blocks.
% That family of codes are named \acp{lrc} \autocite{lrc}.
% Codes like \ac{rs} are \ac{mds};
%they have optimal storage overhead for a given level of fault-tolerance~\autocite{XorbasVLDB}.
%However, they have the worst possible locality.
%\Iac{lrc} is not \ac{mds}, so it requires a little more storage overhead, but provides improved locality.

This paper introduces ErasureBench, a framework to evaluate erasure coding implementations under real conditions. 
ErasureBench exposes a familiar hierarchical file system interface leveraging FUSE.
The architecture is modular and allows to easily plug and configure different erasure codes.
%ErasureBench allows researchers to benchmark parameters of erasure codes. 
%Benchmarks can be virtually any piece of software that can run on Linux, and the interface exposed to the benchmarks is a standard hierarchical file system. 
Our motivation is that testing erasure coding implementations in large distributed systems is a challenging endeavor which, in the absence of being able to run benchmarks on a full-fledged cluster from a large operator like Facebook~\autocite{XorbasVLDB} or Microsoft \cite{DBLP:conf/usenix/HuangSXOCG0Y12}, is usually done using simulations~\autocite{Silberstein2014}. 
In contrast, we target real implementations using a small amount of hardware. 
A unique feature of ErasureBench is its ability to automatically instantiate a cluster of storage nodes that can be dynamically scaled during execution using containerization. % based on Redis\footnote{\url{http://redis.io/}} 
Furthermore, ErasureBench can inject real-world failure trace from the \ac{fta} \autocite{fta-journal} to evaluate a system under realistic conditions. 

Our framework is partially inspired by \autocite{Burihabwa2016}, which mainly targets
the measurement of read/write throughput and data storage overhead. 
They tested algorithms in low-level languages, and their framework exposes a REST API exposed through HTTP.
Instead, ErasureBench offers a file-system interface and the ability to inject failure traces.

%Therefore, we decided to develop a system that can be used to benchmark different erasure code implementations using a reasonable amount of hardware.
% In \autocite{Silberstein2014}, the authors target a lazy block repair strategy and implemented a simulator called \textit{ds-sim}. It models failures in a storage cluster, and can evaluate the costs of repair of different erasure coding algorithm, as well as replication.

%\subsection{Hyfs: Design and Implementation of a Reliable File System}

%A filesystem that uses erasure codes and stores data on multiple nodes is presented in \autocite{hyfs}. It exposes a \ac{fuse} interface to the user, and employs standard NFS servers as its backend. The system has is tested using at most 4 nodes.

%\subsection{Lazy Means Smart: Reducing Repair Bandwidth Costs in Erasure-coded Distributed Storage}

% The article evaluates the costs applying a lazy block repair strategy \autocite{Silberstein2014}.
% Instead of repairing blocks right after any failure, a waiting time is introduced before repairing blocks. In case of a temporary failure, no network capacity is wasted repairing intact blocks.
% As part of their evaluation, 

The rest of the article is organised as follows. 
In Section~\ref{sec:codes}, we briefly introduce erasure coding concepts and the codes used in our evaluation. 
The architecture of ErasureBench is detailed in Section~\ref{sec:erasure-tester}. 
Section~\ref{sec:evaluation} presents our evaluation, comparing RS and LRC codes, before concluding in Section~\ref{sec:conclusion}. 
%\hm{I did not write any result in this section, but they are in the abstract. We could add them here as well if there is room.}
