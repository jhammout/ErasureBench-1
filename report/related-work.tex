\section{Related work}

\subsection{XORing Elephants: Novel Erasure Codes for Big Data}

The article presents a new family of erasure codes called \acp{lrc} \autocite{XorbasVLDB}.
These codes enable local repair of faulty data.
With traditional erasure codes like Reed-Solomon, the cumulative size of the blocks needed to repair a file has to be bigger or equal than the original size of the file.
With \ac{lrc}, a failure affecting a small number of blocks can be repaired using a smaller number of clean blocks.
The authors implemented their algorithm in Hadoop HDFS and deployed a test to Facebook clusters.
They measured that the repair process of \ac{lrc} uses half the disk and network bandwidth compared to Reed-Solomon, at the expense of \SI{14}{\percent} more storage overhead.

\subsection{A Performance Evaluation of Erasure Coding Libraries for Cloud-Based Data Stores}

The authors evaluate the performances of different erasure coding libraries \autocite{Burihabwa2016}.
In order to do the job, they developed a system that is similar to the one presented in this paper.
Their system is mainly geared towards the measurement of read/write throughput and data storage overhead of each erasure coding algorithm.
The algorithms that they tested are implemented in low-level languages.
The interface that their system exposes is a REST API exposed through HTTP.

\subsection{Hyfs: Design and Implementation of a Reliable File System}

The author designed a filesystem that uses erasure codes and stores data on multiple nodes \autocite{hyfs}.
It exposes a \ac{fuse} interface to the user, and employs standard NFS servers as its backend.
The system has been tested using at most 4 nodes.

\subsection{Lazy Means Smart: Reducing Repair Bandwidth Costs in Erasure-coded Distributed Storage}

The article evaluates the costs applying a lazy block repair strategy \autocite{Silberstein2014}.
Instead of repairing blocks right after any failure, a waiting time is introduced before repairing blocks.
In case of a temporary failure, no network capacity is wasted uselessly repairing intact blocks.
As part of their evaluation, the authors implemented a simulator called \textit{ds-sim}.
It models failures in a storage cluster, and can evaluate the costs of repair of different erasure coding algorithm, as well as replication.
