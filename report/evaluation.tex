\section{Evaluation}
\label{sec:evaluation}

\subsection{Fault-tolerance}
\label{subsec:fault-tolerance}

One of the most important, or perhaps the most important characteristic of an erasure coding algorithm is its tolerance to faults.
A good code should ensure that data stays completely available when a small fraction of servers goes down.
We quantified the fault tolerance of \acf{rs}, \acf{lrc} and no erasure coding.
We instantiated a storage cluster of 100 nodes, and extracted the contents of an archive in it.
We used 2 different archives.
The first (called \textit{10 bytes} in the figure) contained 1000 files containing 10 random bytes each.
It is an ideal case scenario.
The second archive was the official binary archive of the \textit{httpd 2.4.18} server downloaded from Apache.
After the files were extracted, we checked the integrity of each file contained in the storage cluster.
We then killed storage nodes one after the other, and checked the files integrity before each step.
\autoref{fig:checksum-plot} shows the availability ratio of individual files with regard to the ratio of dead nodes.

\todo[inline]{Analyze the plot}

\begin{itemize}
    \item Read and write throughput on an intact cluster
    \item Read and write throughput in degraded conditions
    \
\end{itemize}

\begin{figure}
    \centering
    \begin{tikzpicture}
\pgfplotsset{
    height=5cm,
    width=\linewidth,
}
\usetikzlibrary{plotmarks}
\begin{axis}[
    xlabel={Dead nodes $\left[\si{\percent}\right]$},
    ylabel={Available files $\left[\si{\percent}\right]$},
    cycle list name=exotic,
    mark phase=0,
    mark repeat=5,
    ytick={0, 0.25, 0.5, 0.75, 1},
    legend style={
        short line legend,
        draw=none,
        cells={anchor=west},
    },
    scaled ticks=base 10:2,
    xtick scale label code/.code={},
    ytick scale label code/.code={},
]
\addplot table[x=bench-apache-Null-x, y=bench-apache-Null-y] {plots/checksum-100nodes.dat};
\addlegendentry{httpd, Null}
\addplot table[x=bench-10bytes-Null-x, y=bench-10bytes-Null-y] {plots/checksum-100nodes.dat};
\addlegendentry{10bytes, Null}

\addplot table[x=bench-apache-SimpleRegenerating-x, y=bench-apache-SimpleRegenerating-y] {plots/checksum-100nodes.dat};
\addlegendentry{httpd, LRC}
\addplot table[x=bench-10bytes-SimpleRegenerating-x, y=bench-10bytes-SimpleRegenerating-y] {plots/checksum-100nodes.dat};
\addlegendentry{10bytes, LRC}

\addplot table[x=bench-apache-ReedSolomon-x, y=bench-apache-ReedSolomon-y] {plots/checksum-100nodes.dat};
\addlegendentry{httpd, RS}
\addplot table[x=bench-10bytes-ReedSolomon-x, y=bench-10bytes-ReedSolomon-y] {plots/checksum-100nodes.dat};
\addlegendentry{10bytes, RS}
\end{axis}
\end{tikzpicture}

    \caption{Fault tolerance of: no erasure coding (Null), \acl{lrc} $\left(10,6,5\right)$ and \acl{rs} $\left(10,4\right)$. Data written on 100 nodes, and then read after killing each node.}
    \label{fig:checksum-plot}
\end{figure}

\begin{figure*}
    \centering
    %\pgfplotstabletypeset[string type]{plots/websites-size.dat}

\begin{tikzpicture}
\usetikzlibrary{plotmarks}
\pgfplotsset{width=\linewidth, height=7cm,
    every axis plot post/.append style={
        solid,
        thin,
        mark=none
    }
}
\begin{axis}[
    date coordinates in=x,
    xlabel=Date,
    ylabel=Available nodes,
    table/col sep=comma,
    x tick label style={rotate=20}
]
\addplot+ table[x=date, y=size] {plots/websites-size.dat};
\end{axis}
\end{tikzpicture}

    \caption{Graphical representation of the number of nodes available at a given time as recorded in the trace file.}
    \label{fig:trace-plot}
\end{figure*}
